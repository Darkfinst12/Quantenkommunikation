%%%%%%%%%%%%%%%%%%%%%%%%%%%%%%%%%%%%%%%%%%%%%%%%%%%%%%%%%%%%%%
%
% 1. Personendaten
%
\author[Diener, Roob, B{\'e}k{\'e}si]
{Irene~Diener, Toni~Roob, Jarod~A.~M.~B{\'e}k{\'e}si}
\newcommand{\myTitle}{Einf{\"u}hrung in die Quantenkommunikation}
\newcommand{\myTitleShort}{Quantenkommunikation}
\title[\myTitleShort]{\myTitle}
%\subtitle{}
\newcommand{\myProf}{Prof.~Dr.~habil.~Andriy~Luntovskyy}
\newcommand{\myAcademy}{Duale Hochschule Sachsen}
\newcommand{\myStudy}{Studiengang Informationstechnik}
\newcommand{\myLocation}{Dresden} 
\newcommand{\myVersion}{Version 1.0}
% Abgabedatum
\newcommand{\mySubmissionDate}{30. September 2025}


%%%%%%%%%%%%%%%%%%%%%%%%%%%%%%%%%%%%%%%%%%%%%%%%%%%%%%%%%%%%%%
%
% 2. Packete
%
\usepackage[utf8]{inputenc}
\usepackage[T1]{fontenc}
\usepackage{lmodern}

\usepackage{hyphenat}
\usepackage[ngerman,english]{babel}
\usepackage[babel]{csquotes}
\usepackage{cite}
\usepackage{amsmath,amssymb,amstext,amsthm}
\usepackage{upgreek,units}

\usepackage{ragged2e}
\usepackage{animate}

\usepackage{graphicx}
\usepackage{caption,subcaption}

\usepackage[hang,multiple]{footmisc}
\renewcommand{\footnotemargin}{1.2em}
\usepackage{longtable,booktabs,multirow,colortbl}
\usepackage{rotating}
\usepackage{remreset}

\usepackage{enumerate}
\usepackage[shortlabels]{enumitem}
\setlist{noitemsep, itemsep=3pt, topsep=0pt, partopsep=6pt}


%%%%%%%%%%%%%%%%%%%%%%%%%%%%%%%%%%%%%%%%%%%%%%%%%%%%%%%%%%%%%%
%
% 3. Settings
%
\date{\mySubmissionDate}

\logo{\includegraphics[width=2cm]{figures/CD/DHSN-Logo}}
\setbeamercovered{transparent}


%The next block of commands puts the table of contents at the 
%beginning of each section and highlights the current section:
\AtBeginSection[]
{
	\begin{frame}[allowframebreaks]
		\frametitle{Inhaltsverzeichnis}
		\tableofcontents[currentsection]
	\end{frame}
}


%%%%%%%%%%%%%%%%%%%%%%%%%%%%%%%%%%%%%%%%%%%%%%%%%%%%%%%%%%%%%%
%
% 4. Custom Commands
%
\usepackage{pdfpages}
\newcommand{\imgbox}[7]{%
	\begin{wrapfigure}{#1}{#2}%
		\centering%
		\vspace{-0.5em}%
		\includegraphics[width=#3]{figures/#4}%
		\vspace{-1.0em}%
		\setcounter{figure}{\thethmhelper}%
		\addtocounter{thmhelper}{1}%
		\caption[#6]{#7}\label{fig:#5}%
		\vspace{-0.5em}%
	\end{wrapfigure}%
}%
\newcommand{\img}[5]{%
	\begin{figure}[htbp]%
		%% Eine normale Abbildung
		%%  #1 Breite der Abbildung
		%%  #2 Abbildung
		%%  #3 Label
		%%  #4 TOC-Beschriftung
		%%  #5 Beschriftung
		%%
		%% (Falls es komisch aussieht, setzen Sie
		%%  nur „t“ oder „tb“ statt „htbp“)
		\vspace{2.5em}%
		\centering%
		\includegraphics[width=#1]{figures/#2}
		\vspace{-0.5em}%
		\setcounter{figure}{\thethmhelper}%
		\addtocounter{thmhelper}{1}%
		\caption[#4]{#5}\label{fig:#3}%
		\vspace{1.5em}%
	\end{figure}%
}%
\newcommand\imgAB[9]{%
	%% LaTeX-Hexerei mit \imgABcontinued
	%% (wegen mehr als 9 Parametern)
	\def\tmpvalone{#1}%
	\def\tmpvaltwo{#2}%
	\def\tmpvalthr{#3}%
	\def\tmpvalfou{#4}%
	\def\tmpvalfiv{#5}%
	\def\tmpvalsix{#6}%
	\def\tmpvalsev{#7}%
	\def\tmpvaleig{#8}%
	\def\tmpvalnin{#9}%
	\imgABcontinued%
}%
\newcommand{\imgABcontinued}[2]{%
	\begin{figure}[htbp]%
		%% zwei Abbildungen nebeneinander
		%%  #1 Label des Containers
		%%  #2 TOC-Beschriftung des Containers
		%%  #3 Beschriftung des Containers
		%%  #4 Breite des linken Containers (A)
		%%  #5 Breite der Abbildung in A
		%%  #6 Abbildung in A
		%%  #7 Beschriftung in A
		%%  #8 Breite des rechten Containers (B)
		%%  #9 Breite der Abbildung in B
		%% #10 Abbildung in B
		%% #11 Beschriftung in B
		%%
		%% (Falls es komisch aussieht, setzen Sie
		%%  nur „t“ oder „tb“ statt „htbp“)
		\vspace{2.5em}%
		\centering%
		\begin{subfigure}[t]{\tmpvalfou\textwidth}%
			\centering%
			\includegraphics[width=\tmpvalfiv]{figures/\tmpvalsix}%
			\caption{\tmpvalsev}\label{fig:\tmpvalone-A}%
		\end{subfigure}%
		\hfill~%
		\begin{subfigure}[t]{\tmpvaleig\textwidth}%
			\centering%
			\includegraphics[width=\tmpvalnin]{figures/#1}%
			\caption{#2}\label{fig:\tmpvalone-B}%
		\end{subfigure}%
		\vspace{-0.5em}%
		\addtocounter{thmhelper}{1}%
		\setcounter{figure}{\thethmhelper}%
		\caption[\tmpvaltwo]{\tmpvalthr}\label{fig:\tmpvalone}%
		\vspace{1.5em}%
	\end{figure}%
}%
\newcommand{\imgABstack}[9]{%
	\begin{figure}[htbp]%
		%% zwei Abbildungen untereinander
		%% #1 Label des Containers
		%% #2 TOC-Beschriftung des Containers
		%% #3 Beschriftung des Containers
		%% #4 Breite der oberen Abbildung (A)
		%% #5 obere Abbildung (A)
		%% #6 Beschriftung der oberen Abbildung (A)
		%% #7 Breite der unteren Abbildung (B)
		%% #8 untere Abbildung (B)
		%% #9 Beschriftung der unteren Abbildung (B)
		%%
		%% (Falls es komisch aussieht, setzen Sie
		%%  nur „t“ oder „tb“ statt „htbp“)
		\vspace{2.5em}%
		\centering%
		\begin{subfigure}[t]{\textwidth}%
			\centering%
			\includegraphics[width=#4]{figures/#5}%
			\caption{#6}\label{fig:#1-A}%
		\end{subfigure}%
		\\%
		\vspace{2em}%
		\begin{subfigure}[t]{\textwidth}%
			\centering%
			\includegraphics[width=#7]{figures/#8}%
			\caption{#9}\label{fig:#1-B}%
		\end{subfigure}%
		\vspace{-0.5em}%
		\addtocounter{thmhelper}{1}%
		\setcounter{figure}{\thethmhelper}%
		\caption[#2]{#3}\label{fig:#1}%
		\vspace{1.5em}%
	\end{figure}%
}%
\newcommand\imgABC[9]{%
	%% LaTeX-Hexerei mit \imgABCcontinued
	%% (wegen mehr als 9 Parametern)
	\def\tmpvalone{#1}%
	\def\tmpvaltwo{#2}%
	\def\tmpvalthr{#3}%
	\def\tmpvalfou{#4}%
	\def\tmpvalfiv{#5}%
	\def\tmpvalsix{#6}%
	\def\tmpvalsev{#7}%
	\def\tmpvaleig{#8}%
	\def\tmpvalnin{#9}%
	\imgABCcontinued%
}%
\newcommand{\imgABCcontinued}[3]{%
	\begin{figure}[htbp]%
		%% drei Abbildungen nebeneinander
		%% #1 Label des Containers
		%% #2 TOC-Beschriftung des Containers
		%% #3 Beschriftung des Containers
		%% #4 Breite der linken Abbildung (A)
		%% #5 linke Abbildung (A)
		%% #6 Beschriftung der linken Abbildung (A)
		%% #7 Breite der mittleren Abbildung (B)
		%% #8 mittlere Abbildung (B)
		%% #9 Beschriftung der mittleren Abbildung (B)
		%% #10 Breite der rechten Abbildung (C)
		%% #11 rechte Abbildung (C)
		%% #12 Beschriftung der rechten Abbildung (C)
		%%
		%% (Falls es komisch aussieht, setzen Sie
		%%  nur „t“ oder „tb“ statt „htbp“)
		\vspace{2.5em}%
		\centering%
		\begin{subfigure}[t]{0.3\textwidth}%
			\centering%
			\includegraphics[width=\tmpvalfou]{figures/\tmpvalfiv}%
			\caption{\tmpvalsix}\label{fig:\tmpvalone-A}%
		\end{subfigure}%
		\hfill~%
		\begin{subfigure}[t]{0.3\textwidth}%
			\centering%
			\includegraphics[width=\tmpvalsev]{figures/\tmpvaleig}%
			\caption{\tmpvalnin}\label{fig:\tmpvalone-B}%
		\end{subfigure}%
		\hfill~%
		\begin{subfigure}[t]{0.3\textwidth}%
			\centering%
			\includegraphics[width=#1]{figures/#2}%
			\caption{#3}\label{fig:\tmpvalone-C}%
		\end{subfigure}%
		\vspace{-0.5em}%
		\addtocounter{thmhelper}{1}%
		\setcounter{figure}{\thethmhelper}%
		\caption[\tmpvaltwo]{\tmpvalthr}\label{fig:\tmpvalone}%
		\vspace{1.5em}%
	\end{figure}%
}%
\newcommand\imgABCstack[9]{%
	%% LaTeX-Hexerei mit \imgABCcontinued
	%% (wegen mehr als 9 Parametern)
	\def\tmpvalone{#1}%
	\def\tmpvaltwo{#2}%
	\def\tmpvalthr{#3}%
	\def\tmpvalfou{#4}%
	\def\tmpvalfiv{#5}%
	\def\tmpvalsix{#6}%
	\def\tmpvalsev{#7}%
	\def\tmpvaleig{#8}%
	\def\tmpvalnin{#9}%
	\imgABCcontinued%
}%
\newcommand{\imgABCstackcontinued}[3]{%
	\begin{figure}[htbp]%
		%% drei Abbildungen untereinander
		%% #1 Label des Containers
		%% #2 TOC-Beschriftung des Containers
		%% #3 Beschriftung des Containers
		%% #4 Breite der oberen Abbildung (A)
		%% #5 obere Abbildung (A)
		%% #6 Beschriftung der oberen Abbildung (A)
		%% #7 Breite der mittleren Abbildung (B)
		%% #8 mittlere Abbildung (B)
		%% #9 Beschriftung der mittleren Abbildung (B)
		%% #10 Breite der unteren Abbildung (C)
		%% #11 untere Abbildung (C)
		%% #12 Beschriftung der unteren Abbildung (C)
		%%
		%% (Falls es komisch aussieht, setzen Sie
		%%  nur „t“ oder „tb“ statt „htbp“)
		\vspace{2.5em}%
		\centering%
		\begin{subfigure}[t]{\textwidth}%
			\centering%
			\includegraphics[width=\tmpvalfou]{figures/\tmpvalfiv}%
			\caption{\tmpvalsix}\label{fig:\tmpvalone-A}%
		\end{subfigure}%
		\\%
		\vspace{2em}%
		\begin{subfigure}[t]{\textwidth}%
			\centering%
			\includegraphics[width=\tmpvalsev]{figures/\tmpvaleig}%
			\caption{\tmpvalnin}\label{fig:\tmpvalone-B}%
		\end{subfigure}%
		\\%
		\vspace{2em}%
		\begin{subfigure}[t]{\textwidth}%
			\centering%
			\includegraphics[width=#1]{figures/#2}%
			\caption{#3}\label{fig:\tmpvalone-C}%
		\end{subfigure}%
		\vspace{-0.5em}%
		\addtocounter{thmhelper}{1}%
		\setcounter{figure}{\thethmhelper}%
		\caption[\tmpvaltwo]{\tmpvalthr}\label{fig:\tmpvalone}%
		\vspace{1.5em}%
	\end{figure}%
}%
\newcommand\imgABCD[9]{%
	%% LaTeX-Hexerei mit \imgABCcontinued
	%% (wegen mehr als 9 Parametern)
	\def\tmpvalone{#1}%
	\def\tmpvaltwo{#2}%
	\def\tmpvalthr{#3}%
	\def\tmpvalfou{#4}%
	\def\tmpvalfiv{#5}%
	\def\tmpvalsix{#6}%
	\def\tmpvalsev{#7}%
	\def\tmpvaleig{#8}%
	\def\tmpvalnin{#9}%
	\imgABCDcontinued%
}%
\newcommand{\imgABCDcontinued}[6]{%
	\begin{figure}[htbp]%
		%% vier Abbildungen in 2x2-Anordnung (Spalten je 50%)
		%% #1 Label des Containers
		%% #2 TOC-Beschriftung des Containers
		%% #3 Beschriftung des Containers
		%% #4 Breite der oberen, linken Abbildung (A)
		%% #5 obere, linke Abbildung (A)
		%% #6 Beschriftung der oberen, linken Abbildung (A)
		%% #7 Breite der oberen, rechten Abbildung (B)
		%% #8 obere, rechte Abbildung (B)
		%% #9 Beschriftung der oberen, rechten Abbildung (B)
		%% #10 Breite der unteren, linken Abbildung (C)
		%% #11 untere, linke Abbildung (C)
		%% #12 Beschriftung der unteren, linken Abbildung (C)
		%% #13 Breite der unteren, rechten Abbildung (D)
		%% #14 untere, rechten Abbildung (D)
		%% #15 Beschriftung der unteren, rechten Abbildung (D)
		%%
		%% (Falls es komisch aussieht, setzen Sie
		%%  nur „t“ oder „tb“ statt „htbp“)
		\vspace{2.5em}%
		\centering%
		\begin{subfigure}[t]{.45\textwidth}%
			\centering%
			\includegraphics[width=\tmpvalfou]{figures/\tmpvalfiv}%
			\caption{\tmpvalsix}\label{fig:\tmpvalone-A}%
		\end{subfigure}%
		\hfill~%
		\begin{subfigure}[t]{.45\textwidth}%
			\centering%
			\includegraphics[width=\tmpvalsev]{figures/\tmpvaleig}%
			\caption{\tmpvalnin}\label{fig:\tmpvalone-B}%
		\end{subfigure}%
		\\%
		\vspace{2em}%
		\begin{subfigure}[t]{.45\textwidth}%
			\centering%
			\includegraphics[width=#1]{figures/#2}%
			\caption{#3}\label{fig:\tmpvalone-C}%
		\end{subfigure}%
		\hfill~%
		\begin{subfigure}[t]{.45\textwidth}%
			\centering%
			\includegraphics[width=#4]{figures/#5}%
			\caption{#6}\label{fig:\tmpvalone-D}%
		\end{subfigure}%
		\vspace{-0.5em}%
		\addtocounter{thmhelper}{1}%
		\setcounter{figure}{\thethmhelper}%
		\caption[\tmpvaltwo]{\tmpvalthr}\label{fig:\tmpvalone}%
		\vspace{1.5em}%
	\end{figure}%
}%