\section{Quantennetzwerke}
\begin{frame}
%	\frametitle{Quantennetzwerke}
	\textbf{Quantennetzwerke} = klassisches Netzwerk für Quanteninformation\\
	\vspace{2em}
	\textbf{Aufbau:}\\
		\hspace{3.2em} \textbf{Knoten:} Sender, Empf{\"a}nger, Zwischenstationen\\
		\hspace{3.2em} \textbf{Quantenkan{\"a}le:}\\
		\hspace{3.2em} Glasfaser, Satelliten \textrightarrow{ }verschr{\"a}nkte Photonen oder Qubits\\
		\hspace{3.2em} \textbf{Funktionsprinzip:}\\
		\hspace{3.2em} Erzeugung verschr{\"a}nkter Photonenpaare \\
		\hspace{3.2em} Photon A bleibt bei Alice, Photon B \textrightarrow{ }Bob\\
		\hspace{3.2em} Messung von A \textrightarrow{ }instantane Verbindung zu B
\end{frame}
\begin{frame}
	\textbf{Quanten-Repeater:}\\
	Erzeugen neue verschr{\"a}nkte Paare\\
	Verknüpfen diese via Entanglement Swaps \textrightarrow{ }Knoten weit entfernt verschr{\"a}nkt\\
	Verl{\"a}ngern Reichweite durch kontrollierte Verschmelzung\\
	\vspace{1em}
	\textbf{Klassischer Kanal:}\\
	Austausch von Korrekturbits \& Best{\"a}tigungssignalen\\
	\vspace{1em}
	\textbf{Eigenschaften:}\\
	Quantenkan{\"a}le übertragen einzelne Photonen (Polarisation)\\
	\vspace{1em}
	\textbf{No-Cloning-Theorem:}\\
	Quanteninfos nicht kopierbar
\end{frame}
\begin{frame}
	\textbf{Anwendungen:}\\
	Quantenkryptographie (BB84, E91)\\
	Quanten-Teleportation\\
	Verteilte Quantencomputer\\
	Fundamentale Physik-Tests\\
	\vspace{1em}
	\textbf{Ausblick:}\\
	Ziel: Quanteninternet \textrightarrow{ }verschr{\"a}nkte Zust{\"a}nde weltweit verteilen\\
	Erste Demos: Glasfaser \& Satelliten (z. B. \enquote{Micius}-Projekt)\\
	Zukunft: Erg{\"a}nzung/Absicherung klassischer Kommunikationsnetze
\end{frame}