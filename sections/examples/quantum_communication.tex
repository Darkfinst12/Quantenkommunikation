\section{Quantenkommunikation an Hand eines Beispiels}
\begin{frame}[allowframebreaks]
	
\end{frame}

\subsection{{\"U}bungsaufgabe I}
\begin{frame}[allowframebreaks]
	\begin{block}{Aufgabe}
		Bestimmen Sie die Wahrscheinlichkeit eines horizontal polarisierten Photons in die jeweiligen Zust{\"a}nde der D/A-Basis bzw. der H/V-Basis zu fallen.
	\end{block}
\end{frame}
\begin{frame}[allowframebreaks]
	\begin{block}{L{\"o}sung}
		\begin{alignat*}{3}
			P(D) &= \cos(45)^2 &= 0.5 &\Rightarrow 50\% \\
			P(A) &= \cos(-45)^2 &= 0.5 &\Rightarrow 50\% \\
%		\end{alignat*}
			\\
%		\begin{alignat*}{3}
			P(V) &= \sin(90)^2 &= 0.5 &\Rightarrow 0\% \\
			P(H) &= \sin(0)^2 &= 0.5 &\Rightarrow 100\%
		\end{alignat*}
	\end{block}
\end{frame}
