\section{Was ist Quantenkommunikation?}
\begin{frame}
	\begin{theorem}
%		\justifying
		Quantenkommunikation ist die Nutzung der \enquote{Prinzipien der Quantenmechanik wie Quantenverschr{\"a}nkung und Quantensuperposition, um Informationen nahezu abh{\"o}rsicher zu {\"u}bertragen}.\cite{frauenhofer2025}
	\end{theorem}
	\pause
	-- Nutzung von \textbf{Quantenzust{\"a}nden} (erzeugt durch Polarisation) zur\\
		\hspace{0.5em} abh{\"o}rsicheren Nachrichten{\"u}bertragung\\
	\vspace{0.5em}
	-- Nutzt Superposition und Verschr{\"a}nkung\\
	\vspace{0.5em}
	-- Quantenschl{\"u}sselverteilung zur Erstellung eines gemeinsamen Schl{\"u}ssels\\ 
	\vspace{0.5em}
		\hspace{0.5em} (siehe Protokolle der Kryptographie)\\
	-- Besondere Eigenschaft: Abh{\"o}ren ver{\"a}ndert automatisch den\\ 
		\hspace{0.5em} Quantenzustand \textrightarrow erkennbar
\end{frame}

\subsection{Polarisation}
\begin{frame}
	\begin{theorem}
		Licht besteht aus elektromagnetischen Wellen. Das elektrische Feld schwingt immer senkrecht zur Ausbreitungsrichtung. Die Richtung dieser Schwingung nennt man \textbf{Polarisation}.
	\end{theorem}
\end{frame}
\begin{frame}[allowframebreaks]
	\frametitle{Polarisationsrichtungen$^1$}
	\framesubtitle{H ,V ,D ,A}
%	\vspace{-2em}
	\begin{alertblock}{Basen (Photonenzust{\"a}nde):}
		Orthogonal: H/V – Basis (Z-Basis)\\
			\hspace{0.5em}  H, $0^\circ$: horizontal \textrightarrow\\
			\hspace{0.5em}  V, $90^\circ$: vertikal  \textuparrow\\
		Schr{\"a}g: D/A – Basis (X-Basis)\\
			\hspace{0.5em}  D, $45^\circ$: diagonal $\nearrow$\\
			\hspace{0.5em}  A, $135^\circ$: antidiagonal $\nwarrow$\\
	\end{alertblock}
	\begin{myExamples}{Zustände werden als |0⟩ oder |1⟩ festgelegt}
		z.B: H/V -- Basis \textrightarrow { }$|H\rangle = 0$, $|V\rangle = 1$
	\end{myExamples}
	\footnotetext[1]{lineare Polarisation}
\end{frame}

\begin{frame}
	\frametitle{Wie funktioniert sie?}
	-- Erfordert Basiselemente des Quantencomputers\\
	-- Polarisation der Schwingungsrichtungen der Photonen mittels\\
		\hspace{0.5em} Polarisationsfilter \textrightarrow Erzeugung von Qubits\\
	\vspace{-2.5em}
	\hspace{9em}\animategraphics[autoplay,loop,width=0.7\linewidth]{24}{figures/Kommunikation/Polarisation/polarisation-frame-}{00}{24}\\
	\vspace{-2.0em}
	-- Begrenzung: Photonenabsorption in Glasfasern \textrightarrow { }ca. 100 km\\
		\hspace{0.5em} Reichweite\\
	-- L{\"o}sung: Quantenrepeater zur Reichweitenerhöhung \textrightarrow { }zentrales\\
	\hspace{0.5em} Forschungsthema
\end{frame}

