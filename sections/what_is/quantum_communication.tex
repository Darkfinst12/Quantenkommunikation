\section{Was ist Quantenkommunikation?}
\begin{frame}
	\begin{Definition}
%		\justifying
		Quantenkommunikation ist die Nutzung der \enquote{Prinzipien der Quantenmechanik wie Quantenverschr{\"a}nkung und Quantensuperposition, um Informationen nahezu abh{\"o}rsicher zu {\"u}bertragen}.\cite{frauenhofer2025}
	\end{Definition}
	\pause
	-- Nutzung von \textbf{Quantenzust{\"a}nden} (erzeugt durch Polarisation) zur\\
		\hspace{0.5em} abh{\"o}rsicheren Nachrichten{\"u}bertragung\\
	\vspace{0.5em}
	-- Nutzt Superposition und Verschr{\"a}nkung\\
	\vspace{0.5em}
	-- Quantenschl{\"u}sselverteilung zur Erstellung eines gemeinsamen Schl{\"u}ssels\\ 
	\vspace{0.5em}
		\hspace{0.5em} (siehe Protokolle der Kryptographie)\\
	-- Besondere Eigenschaft: Abh{\"o}ren ver{\"a}ndert automatisch den\\ 
		\hspace{0.5em} Quantenzustand \textrightarrow { }erkennbar
\end{frame}

\subsection{Polarisation}
\begin{frame}
	\begin{Definition}
		Licht besteht aus elektromagnetischen Wellen. Das elektrische Feld schwingt immer senkrecht zur Ausbreitungsrichtung. Die Richtung dieser Schwingung nennt man \textbf{Polarisation}.
	\end{Definition}
\end{frame}
\begin{frame}[allowframebreaks]
	\frametitle{Polarisationsrichtungen$^1$}
	\framesubtitle{H ,V ,D ,A}
	\footnotetext[1]{lineare Polarisation}
%	\vspace{-2em}
	\begin{alertblock}{Basen (Photonenzust{\"a}nde):}
		Orthogonal: H/V – Z-Basis\\
			\hspace{0.5em}  H, $0^\circ$: horizontal \textrightarrow\\
			\hspace{0.5em}  V, $90^\circ$: vertikal  \textuparrow\\
		Schr{\"a}g: D/A – X-Basis\\
			\hspace{0.5em}  D, $45^\circ$: diagonal $\nearrow$\\
			\hspace{0.5em}  A, $135^\circ$: antidiagonal $\nwarrow$\\
	\end{alertblock}
	\begin{alertblock}{Jones-Vektoren}
		\[
		|H\rangle 
		= 
		\left(\begin{array}{c}
			1 \\ 0
		\end{array}\right),
		\quad
		|V\rangle 
		= 
		\left(\begin{array}{c}
			0 \\ 1
		\end{array}\right),
		\]
		\[
		|D\rangle 
		= \frac{1}{\sqrt{2}} 
		\left(\begin{array}{c}
		1 \\ 1
		\end{array}\right),
		\quad
		|A\rangle 
		= \frac{1}{\sqrt{2}} 
		\left(\begin{array}{c}
		1 \\ -1
		\end{array}\right)
		\]
	\end{alertblock}
	\begin{myExamples}{Zustände werden als |0⟩ oder |1⟩ festgelegt}
		z.B: H/V -- Basis \textrightarrow { }$|H\rangle = 0$, $|V\rangle = 1$
	\end{myExamples}
	
\end{frame}

\begin{frame}
	\frametitle{Wie funktioniert sie?}
	-- Erfordert Basiselemente des Quantencomputers\\
	-- Polarisation der Schwingungsrichtungen der Photonen mittels\\
		\hspace{0.5em} Polarisationsfilter \textrightarrow { }Erzeugung von Qubits\\
	\vspace{-2.5em}
	\hspace{9em}\animategraphics[autoplay,loop,width=0.7\linewidth]{24}{figures/Kommunikation/Polarisation/polarisation-frame-}{00}{24}\\
	\vspace{-2.0em}
	-- Begrenzung: Photonenabsorption in Glasfasern \textrightarrow { }ca. 100 km\\
		\hspace{0.5em} Reichweite\\
	-- L{\"o}sung: Quantenrepeater zur Reichweitenerhöhung \textrightarrow { }zentrales\\
	\hspace{0.5em} Forschungsthema
\end{frame}

\begin{frame}
	\frametitle{Malus’ Gesetz}
	\begin{Definition}
		\[\mathit{P}(\text{Durchgang}) = \cos^2(\phi - \theta)\]
		$\phi$: Polarisationswinkel; $\theta$: Ausrichtung des Polarisators 
	\end{Definition}
	-- \textbf{Klassisch}: Dieses Gesetz beschreibt die \textit{Intensi{\"a}t} des Lichtstrahls nach\\
		\hspace{0.5em} einem Polarisator.\\
	-- \textbf{Quantenmechanisch}: Es ist die \textit{Wahrscheinlichkeit}, dass ein einzelnes\\
		\hspace{0.5em} Photon durchkommt.
\end{frame}

\begin{frame}
	\frametitle{{\"U}bung I}
	Ein horizontal polarisiertes Photon ($\phi = 0^\circ$) trifft auf einen Polarisator, 
	dessen Transmissionsrichtung bei $\theta = 45^\circ$ liegt.\\
	Wie groß ist die Wahrscheinlichkeit, dass das Photon den Polarisator passiert?
\end{frame}

\begin{frame}
	\frametitle{L{\"o}sung I}
	\framesubtitle{Photon horizontal}
	$\phi = 0^\circ$; $\theta = 45^\circ$
	\[
		\mathit{P} = \cos^2(-45^\circ) = 0.5
	\]
	$\Rightarrow$ Jedes Photon hat also eine $50\%$ Chance, durchzukommen.
\end{frame}

\begin{frame}
	\frametitle{{\"U}bung II}
	Ein Laserstrahl mit einer Intensität von $I_0 = 10 \,\text{mW}$ ist bei $30^\circ$ polarisiert. Er trifft auf einen Polarisator, dessen Transmissionsrichtung bei $0^\circ$ liegt.\\
	Welche Intensität $I$ hat der Laserstrahl nach dem Durchgang durch den Polarisator?
\end{frame}

\begin{frame}
	\frametitle{L{\"o}sung II}
	\framesubtitle{Laserlicht}
	$I_0 = 10 mW$; Polarisation bei $30^\circ$; Treffen des Polarisators bei $0^\circ$
	\[
		I = I_0 \cos^2(30^\circ) = 7.5 mW
	\]
\end{frame}

\begin{frame}
	\frametitle{{\"U}bung III}
	Zwei Polarisatoren sind gekreuzt ($0^\circ$ und $90^\circ$). 
	Es fällt kein Licht durch. \\
	Was passiert jedoch, wenn ein dritter Polarisator mit $45^\circ$ zwischen die beiden eingefügt wird? 
	Wie groß ist dann die Intensität des durchgelassenen Lichts in Abhängigkeit von $I_0$?
\end{frame}


\begin{frame}
	\frametitle{L{\"o}sung III}
	\framesubtitle{Drei-Polarisatoren-Experiment}
	 -- Zwei gekreuzte Polarisatoren ($0^\circ$ und $90^\circ$)\\
	 \hspace{0.5em} $\Rightarrow I = 0$ \\
	 -- Hinzufügen eines dritten Polarisator bei $45^\circ$:\\
	 \[
	 	I = I_0 \cdot \cos^2(45^\circ) \cdot \cos^2(45^\circ) = \frac{1}{4}I_0 
	 \]
	 $\Rightarrow 25\%$ des Lichts passiert den Aufbau.
\end{frame}

\begin{frame}
	\frametitle{Bedeutung in der Quantenkommunikation}
	--  Polarisationszustände sind Träger von Information (Qubits).
\end{frame}